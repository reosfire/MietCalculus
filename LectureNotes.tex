\documentclass[11pt]{article}
\usepackage[T1,T2A]{fontenc}
\usepackage[utf8]{inputenc}

\usepackage[english, russian]{babel}
\usepackage{cmap}
\usepackage{amsmath}
\usepackage{amssymb}
\usepackage{mathabx}
\usepackage{setspace}
\usepackage[left=2cm,right=2cm,top=2cm,bottom=2cm,bindingoffset=0cm]{geometry}
\onehalfspacing

\title{CalculusExam}
\author{reosfire}

\begin{document}
    \section{Определённый интеграл}
        \subsection{Определение}
        Пусть $f(x)$ определена на $[a; b]$, введём:
        \begin{enumerate}
            \item $\rho$ - разбиение $[a; b]$ - последовательность $x_0, x_1, ..., x_{n - 1}, x_n$ такая, что:\\
             $x_0 = a, x_n = b$\\
             $x_0 < x_1 < ... < x_{n - 1} < x_n$
            \item $\Delta x_i = x_{i + 1} - x_i \qquad i \in[0; n - 1]$
            \item $\lambda \rho = max\{\Delta x_i\}$ - мелкость разбиения
            \item $c_i \in [x_i; x_i + 1]$ - произвольная точка отрезка разбиения
        \end{enumerate}
        Тогда определим интегральную сумму:
        $S(\rho, \overline{c}) = \sum_{i=0}^{n-1} f(c_i)*\Delta x_i$\\
        И если $\exists \lim_{\lambda \rho \to 0} S(\rho, \overline{c}) = I$,
        тогда $f(x)$ - интегрируема на $[a; b]$, $I$ - интеграл, и обозначается: \[\int_{a}^{b}f(x)dx\]
        Распишем предел из определения по Коши:
        \[\forall \varepsilon > 0 \quad \exists \delta > 0 \quad \forall \rho, \overline{c} \quad \lambda\rho < \delta \implies |S(\rho, \overline{c}) - I| < \varepsilon\]
        И по Гейне:
        \[\forall \{\rho_k\}, \{\overline{c}_k\} \quad \lambda\rho \to 0 \implies \lim_{k \to \infty} S(\rho_k, \overline{c}_k) = I\]

        \subsection{Необходимое условие интегрируемости}
        Теорема: $f(x)$ интегрируема на $[a; b] \quad \implies f(x)$ - ограничена на $[a; b]$\\
        Док-во(от противного):

        Пусть $f(x)$ неограничена на $[a; b]$ тогда при любом $\rho$ существует такое $k$, что f(x) неограничена на $[x_k; x_{k + 1}]$.
        И т.к выбор точек произвольный: $\forall \rho \quad \forall C > 0 \quad \exists x \in [x_k; x_{k + 1}] \quad |f(x)| > C$

        Хотим получить противоречие с интегрируемостью, т.е предел интегральных сумм не должен существовать. Мы добьёмся этого если:
        $\forall \rho \quad \forall M > 0 \quad \exists \overline{c} \quad \left\vert \sum_{i=0}^{n-1} f(c_i)\Delta x_i \right\vert > M$\\
        $\left\vert \sum_{i=0 i\neq k}^{n-1} f(c_i)\Delta x_i + f(c_k)\Delta x_k \right\vert > M$\\
        $\left\vert \sum_{i=0 i\neq k}^{n-1} f(c_i)\Delta x_i \right\vert + |f(c_k)|\Delta x_k > M$\\
        $|f(c_k)| > \frac{M - \left\vert \sum_{i=0 i\neq k}^{n-1} f(c_i)\Delta x_i \right\vert}{\Delta x_k}$\\
        Зафиксируем все $c_i$ кроме $c_k$, и получим что нам нужно доказать, что $|f(c_k)|$ можно сделать больше любого заданного числа,
         что мы и сделали вначале.

        \subsection{Суммы Дарбу}
        \subsubsection{Определение}
        Пусть $f(x)$ ограничена на $[a; b]$\\
        Введём разбиение $\rho$ и его мелкость $\lambda \rho$ так же, как и в определении интеграла.\\
        Т.к $f(x)$ ограничена на $[a; b]$, то она ограничена на всех её подотрезках(от противного),
        а значит имеет $\sup$ и $\inf$ на каждом из них.\\
        Обозначим
        $\begin{array}{ll}
            \sup(f(x)) = M_i\\
            \inf(f(x)) = m_i
        \end{array}$ \quad
        где $x \in [x_i; x_{i + 1}] \quad i \in [0; n - 1]$\\
        Тогда
        $\begin{array}{ll}
            \overline{S} = \sum_{i = 0}^{n - 1} M_i \Delta x_i \quad $ - верхняя сумма Дарбу$\\
            \underline{S} = \sum_{i = 0}^{n - 1} m_i \Delta x_i \quad $ - нижняя сумма Дарбу$\\
        \end{array}$

        \subsubsection{Свойства}
        \begin{enumerate}
            \item $\forall \rho, \overline{c} \quad \underline{S}(\rho) \leq S(\rho, \overline{c}) \leq \overline{S}(\rho)$
            
            Док-во:\\
            Рассмотрим $[x_i; x_{i + 1}]$\\
            $x_i \leq c_i \leq x_{i + 1}$ \quad воспользуемся непрерывностью $f(x)$\\
            $m_i \leq f(c_i) \leq M_i$ \quad домножим на $\Delta x_i$\\
            $m_i\Delta x_i \leq f(c_i)\Delta x_i \leq M_i\Delta x_i$ \quad просуммируем по всем $i$\\
            $\underline{S}(\rho) \leq S(\rho, \overline{c}) \leq \overline{S}(\rho)$

            \item Добавление точки в $\rho$ не увеличивает $\overline{S}$, не уменьшает $\underline{S}$\\
            $\underline{S}(\rho) \leq \underline{S}(\rho^*) \leq \overline{S}(\rho^*) \leq \overline{S}(\rho)$

            Док-во(для $\overline{S}$):\\
            Пусть мы добавили точку $x^*$ в отрезок $[x_i; x_{i+1}]$, получим два новых отрезка $[x_i; x^*],\ [x^*; x_{i + 1}]$\\
            Обозначим $M^*_1 = \sup(f(x))\ x \in [x_i; x^*] \quad M^*_2 = \sup(f(x))\ x \in [x^*; x_{i + 1}]$\\
            $\Delta_1 = x^* - x_i \quad \Delta_2 = x_{i + 1} - x^*$\\
            $\Delta_1 + \Delta_2 = x^* - x_i + x_{i + 1} - x^* = x_{i + 1} - x_i = \Delta x_i$\\
            Т.к супремум на подотрезке не больше супремума на всём отрезке:\\
            $M_i\Delta_1 + M\Delta_2 \geq M^*_1\Delta_1 + M^*_2\Delta_2$\\
            $M_i(\Delta_1 + \Delta_2) \geq M^*_1\Delta_1 + M^*_2\Delta_2$\\
            $M_i\Delta x_i \geq M^*_1\Delta_1 + M^*_2\Delta_2$\\
            Т.к остальные $M_j \Delta x_j$ у сумм совпадают, то и $\overline{S}(\rho) \geq \overline{S}(\rho^*)$

            \item Если $\rho^*$ получена из $\rho$ добавлением конечного числа точек, выполняются те же неравенства.
            $\underline{S}(\rho) \leq \underline{S}(\rho^*) \leq \overline{S}(\rho^*) \leq \overline{S}(\rho)$

            Док-во - простейшая индукция.

            \item $\forall \rho_1, \rho_2 \quad \underline{S}(\rho_1) \leq \overline{S}(\rho_2)$
            
            Док-во:\\
            Рассмотрим разбиение $\rho_3$, полученное добавлением всех точек из $\rho_1$ и $\rho_2$
            \begin{itemize}
                \item $\underline{S}\rho_1 \leq \underline{S}\rho_3$\quad - по свойству 3 ($\rho_1 = \rho,\ \rho_3 = \rho^*$)
                \item $\underline{S}\rho_3 \leq \overline{S}\rho_3$\quad - по свойству 1
                \item $\overline{S}\rho_3 \leq \overline{S}\rho_2$\quad - по свойству 3 ($\rho_2 = \rho,\ \rho_3 = \rho^*$)
            \end{itemize}
            
            \item Пусть $\rho_0$ - фиксированное, тогда\\
             $\forall \rho \begin{array}{lllll}
                \underline{S}(\rho) \leq \overline{S}(\rho_0) &\implies&
                 \underline{S}(\rho)\ $ограничена сверху$ &\implies&
                 \exists\ \sup(\underline{S}(\rho)) = \underline{I}\\

                \overline{S}(\rho) \geq \underline{S}(\rho_0) &\implies&
                \overline{S}(\rho)\ $ограничена снизу$ &\implies&
                \exists\ \inf(\overline{S}(\rho)) = \overline{I}
            \end{array}$
        \end{enumerate}

        \subsection{Критерий интегрируемости}
        Пусть $f(x)$ - ограничена на $[a; b]$, тогда равносильны следующие утверждения:
        \begin{enumerate}
            \item $\lim_{\lambda \rho \to 0} (\overline{S} - \underline{S}) = 0$
            \item $\forall \varepsilon > 0 \quad \exists \rho \quad \overline{S} - \underline{S} < \varepsilon$
            \item $\overline{I} = \underline{I}$
            \item $f(x)$ интегрируема на $[a; b]$ 
        \end{enumerate}
        $\int_{a}^{b} f(x)dx = \overline{I} = \underline{I} = \lim_{\lambda \rho \to 0} (\overline{S}) = \lim_{\lambda \rho \to 0} (\underline{S})$

        \subsection{Определение площади / геометрический смысл определённого интеграла}
        Определим прямоугольник $Rect(a, b, c, d) = \left\{ (x, y) \in \mathbb{R}^2 \left\vert\begin{array}{l}
            x \in [a; b]\\
            y \in [c; d]
        \end{array}\right. \right\}$\\
        И определим его площадь $S(Rect(a, b, c, d)) = (b - a)(d - c)$\\
        Пусть $\sigma = \lor_{i = 1}^{n} Rect_i$ - множество точек, 
        полученное объединением $n$ прямоугольников, пересекающихся не больше, чем по стороне.
        Тогда $S(\sigma) = \sum_{i=1}^{n} S(Rect_i)$

        Пусть дана фигура (произвольное множество точек) $A$.\\
        Пусть существует прямоугольник $B$ такой, что $A \subset B$, если нет, то и площади у такой фигуры нет.

        Рассмотрим $\sigma^+$ - такая $\sigma$, что $A \subset \sigma^+$. Она всегда существует (как минимум $B$)\\
        Заметим, что $S(\sigma^+) \geq 0$ (как и от любой другой $\sigma$),
         значит $\exists\ \inf(S(\sigma^+)) = S^+(A)$ - определение верхней площади.

        Теперь рассмотрим $\sigma^-$ - такая $\sigma$, что $\sigma^- \subset A$. Она может и не существовать(ф-ия Дирихле)
         тогда считаем, что площади у фигуры нет.\\
        Но если существует, то $S(\sigma^-) \leq S(B)$ т.к $\sigma^- \subset A \subset B$, 
         значит $\exists \sup(S(\sigma^-)) = S^-(A)$ - определение нижней площади.

        Если $S^+(A) = S^-(A)\ =\ S(A)$, то $S(A)$ - площадь $A$.

        Определим криволинейную трапецию: $f$ - положительно определена на $[a, b]$\\ $Tr(a, b, f) = \left\{ (x, y) \in \mathbb{R}^2 \left\vert\begin{array}{l}
            x \in [a; b]\\
            y \in [0; f(x)]
        \end{array}\right. \right\}$\\
        Тогда\quad $\exists\ S(Tr(a, b, f)) \iff f(x)$ интегрируема на $[a; b]$\\
        $S(Tr(a, b, f)) = \int_{a}^{b} f(x)dx$\\
        $S(\sigma^+) = \overline{S}(\rho)$\\
        $S(\sigma^-) = \underline{S}(\rho)$

        \subsection{Свойства определённого интеграла}
        \begin{enumerate}
            \item Аддитивность.\\
            Если $f(x)$ - инт на $[a; b] \implies \forall c \in [a; b] \quad f(x) - $ инт на $[a; c]$ и $[c; b]$,\quad и наоборот.
            \[\int_{a}^{b}f(x)dx = \int_{a}^{c}f(x)dx + \int_{c}^{b}f(x)dx\]
            Док-во:\\
            Пользуемся тем, что\quad
            $\forall \varepsilon > 0 \quad \exists \rho \quad \overline{S}(\rho) - \underline{S}(\rho) < \varepsilon \iff f(x) -$ инт на $[a; b]$\\

            $(\implies)$
            Дано: $f(x) -$ инт на $[a; b]$ \quad т.е \quad
            $\forall \varepsilon > 0 \quad \exists \rho \quad \overline{S}(\rho) - \underline{S}(\rho) < \varepsilon$\\
            Пусть $\rho^*$ - разбиение, полученное добавлением точки $c$ в $\rho$, тогда:\\
            $\overline{S}(\rho^*) \leq \overline{S}(\rho)$, \quad $\underline{S}(\rho^*) \geq \underline{S}(\rho)$\\
            Рассмотрим разбиение $\rho_1$, полученное из $\rho^*$ выкидыванием всех точек, не входящих в $[a; c]$.
            Получим: $\overline{S}(\rho_1) - \underline{S}(\rho_1) \leq \overline{S}(\rho^*) - \underline{S}(\rho^*) < \varepsilon$
            
            Это верно потому, что разность между соответствующими элементами верхней и нижней суммами дарбу всегда неотрицательная.
            И т.к $\rho^*$ получается из $\rho_1$ добавлением точек, лежащих вне отрезка, в котором лежат все точки $\rho^*$, 
            то если мы вычтем из правой части неравенства левую, то получим некоторое число неотрицательных слогаемых.\\
            $(\impliedby)$ Дано: $\forall \varepsilon > 0 \quad \exists \rho_1, \rho_2 \quad \begin{array}{l}
                \overline{S}(\rho_1) - \underline{S}(\rho_1) < \frac{\varepsilon}{2} \\
                \overline{S}(\rho_2) - \underline{S}(\rho_2) < \frac{\varepsilon}{2} \\
            \end{array}$,
            где $\rho_1, \rho_ 2$ - разбиения $[a; c], [c; b]$.\\
            Сложим неравенства: $\overline{S}(\rho_1) - \underline{S}(\rho_1) + \overline{S}(\rho_2) - \underline{S}(\rho_2) < \varepsilon$\\
            $\overline{S}(\rho_1) + \overline{S}(\rho_2) - (\underline{S}(\rho_1) + \underline{S}(\rho_2)) < \varepsilon$\\
            $\overline{S}(\rho) - \underline{S}(\rho) < \varepsilon$\\
            
            Определение: если $a > b$, то $\int_{a}^{b}f(x)dx = -\int_{b}^{a}f(x)dx$
            \item Однородность.\\
            
            \item Произведение.\\
            
            \item Переход к неравенству.\\
            
            \item Модуль.\\
            
        \end{enumerate}

        \subsection{Теорема о среднем}

        \subsection{Достаточные признаки интегрируемости}
        \begin{enumerate}
            \item Непрерывные
            
            \item Монотонные
            
        \end{enumerate}

        \subsection{Интеграл как функция верхнего предела}

\end{document}